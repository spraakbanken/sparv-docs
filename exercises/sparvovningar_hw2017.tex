%%%%%%%%%%%%%%%%%%%%%%%%%%%%%%%%%%%%%%%%%
% Short Sectioned Assignment
% LaTeX Template
% Version 1.0 (5/5/12)
%
% This template has been downloaded from:
% http://www.LaTeXTemplates.com
%
% Original author:
% Frits Wenneker (http://www.howtotex.com)
%
% License:
% CC BY-NC-SA 3.0 (http://creativecommons.org/licenses/by-nc-sa/3.0/)
%
%%%%%%%%%%%%%%%%%%%%%%%%%%%%%%%%%%%%%%%%%

%----------------------------------------------------------------------------------------
%	PACKAGES AND OTHER DOCUMENT CONFIGURATIONS
%----------------------------------------------------------------------------------------

\documentclass[paper=a4, fontsize=12pt]{scrartcl} % A4 paper and 11pt font size

\usepackage[T1]{fontenc} % Use 8-bit encoding that has 256 glyphs
\usepackage{fourier} % Use the Adobe Utopia font for the document - comment this line to return to the LaTeX default
\usepackage[swedish]{babel} % SWEDISH language/hyphenation
\usepackage{amsmath,amsfonts,amsthm} % Math packages
\usepackage[utf8]{inputenc} % Math packages
\usepackage{graphicx}
\usepackage[hidelinks]{hyperref}
\usepackage{color}
\usepackage{inconsolata}

\usepackage{lipsum} % Used for inserting dummy 'Lorem ipsum' text into the template
\usepackage{titling}

\usepackage{scrextend} % For use of addmargin

\usepackage[figurename=]{caption}
\usepackage{sectsty} % Allows customizing section commands
\allsectionsfont{\centering \normalfont\scshape} % Make all sections centered, the default font and small caps

\usepackage{fancyhdr} % Custom headers and footers
\pagestyle{fancyplain} % Makes all pages in the document conform to the custom headers and footers
\fancyhead{} % No page header - if you want one, create it in the same way as the footers below
\fancyfoot[L]{} % Empty left footer
\fancyfoot[C]{} % Empty center footer
\fancyfoot[R]{\thepage} % Page numbering for right footer
\renewcommand{\headrulewidth}{0pt} % Remove header underlines
\renewcommand{\footrulewidth}{0pt} % Remove footer underlines
\setlength{\headheight}{13.6pt} % Customize the height of the header

\numberwithin{equation}{section} % Number equations within sections (i.e. 1.1, 1.2, 2.1, 2.2 instead of 1, 2, 3, 4)
\numberwithin{figure}{section} % Number figures within sections (i.e. 1.1, 1.2, 2.1, 2.2 instead of 1, 2, 3, 4)
\numberwithin{table}{section} % Number tables within sections (i.e. 1.1, 1.2, 2.1, 2.2 instead of 1, 2, 3, 4)

\setlength\parskip{0.5\baselineskip}
\setlength\parindent{0pt} % Removes all indentation from paragraphs - comment this line for an assignment with lots of text

\newcommand\todo[1]{{\color{red}#1}}

%----------------------------------------------------------------------------------------
%	TITLE SECTION
%----------------------------------------------------------------------------------------

\newcommand{\horrule}[1]{\rule{\linewidth}{#1}} % Create horizontal rule command with 1 argument of height

\title{\huge Sparv}

% \author{Anne Schumacher}

\hyphenation{kor-pusarna}
\hyphenation{an-ting-en}
\hyphenation{ana-lysen}

\begin{document}

%\maketitle % Print the title
\begin{titlingpage}
\begin{center}
\horrule{0.5pt} \\[0.4cm] % Thin top horizontal rule

\thetitle\\
\vspace{5mm}
\normalsize Övningar\\ Språkbankens höstworkshop 2017
\horrule{2pt} \\[0.5cm] % Thick bottom horizontal rule
\vspace{5mm}
\url{https://spraakbanken.gu.se/sparv}\\
\url{sb-sparv@svenska.gu.se}\\
\vspace{5mm}
17 oktober 2017\\
\vspace{4cm}
\includegraphics[height=4cm,keepaspectratio]{sparv.png}

\end{center}
\end{titlingpage}

\newpage

%----------------------------------------------------------------------------------------
%	PROBLEM 1
%----------------------------------------------------------------------------------------

\section*{Översikt}
Sparv är Språkbankens annoteringsverktyg som används bland annat för att
analysera korpusarna i Korp och texterna i Strix. Sparvs webbgränssnitt
(\url{https://spraakbanken.gu.se/sparv}) kan användas för att annotera egna
texter.

För att göra en analys räcker det att skriva eller klistra in en text i
textfältet och sedan trycka på knappen \textbf{Kör}. När analysen är klar % \verb|Kör|
visas den som tabell under inmatningsfältet. Varje kolumn i tabellen
representerar en viss typ av analys. Man kan få en kort förklaring för
en analystyp genom att hålla musen över namnet i tabellhuvudet.


\section*{Övning 1 - Modern svenska}

Använd Sparv för att analysera meningen:

\newcommand{\example}[1]{\indent\qquad\quad\emph{#1}}

\example{Vad har vi för olika analyser?}

\begin{itemize}
    \item [\textbf{1.1}]
        I kolumnen \textbf{msd} finns den morfologiska beskrivningen.
        Låt muspekaren vila på en förkortning för att få en förklaring.
        Vad har ordet \emph{Vad} fått för beskrivning?

    \item [\textbf{1.2}]
        I \textbf{lemma}-kolumnen kan vi se ordet normaliserat
        till grundform. Håller du med om normaliseringarna?
        Har något ordpar fått en gemensam analys?

    \item [\textbf{1.3}]
        Under \textbf{lex} kan lemgrammen slås upp i Karp och
        under \textbf{sense} ges ordbetydelser från SALDO samt
        sannolikhetsvärden i fall det finns flera analyser per ord.
        Vilka olika uppslag får vi för \emph{ha} och \emph{för}?
\end{itemize}

\noindent
Prova att göra en Sparvanalys av meningen:

\example{Den gamla damen träffade killen med handväskan.}

\begin{itemize}
    \item [\textbf{1.4}]
        Sparv analyserar dependensträdet med hjälp av parsern Malt tränad på Talbanken.
        Vem är det som har handväskan i den här meningen enligt analysen?
        Förklaringar visas när muspekaren är över de olika förkortningarna i trädet.
    
    \item [\textbf{1.5}]
    	Vilken betydelse av \emph{träffa} har fått det högsta sannolikhetsvärdet?
    	(Detta visas i \textbf{sense}-kolumnen.)

    \item [\textbf{1.6}]
        Prova lite andra variationer av meningen för att se om
        handväskan kan analyseras att tillhöra någon annan. 
        Kolla även om betydelserna av ordet \emph{träffa} kan få andra 
        sannolikhetsvärden eller en annan rakning.

    \item [\textbf{1.7}]
        Under \textbf{complemgram} (sammansatta lemgram) och \textbf{compwf}
        (sammansatta ordformer) ses resultatet
        av sammansättningsanalysen. Ett av orden har fått en bra analys,
        men ett av dem en lite sämre. Vad kan det bero på?
\end{itemize}

\pagebreak
\noindent
Klicka i kryssrutan \textbf{Namntaggare} och gör en Sparvanalys av följande mening:

\begin{addmargin}[4em]{4em}
\textit{Den brittiske författaren Kazuo Ishiguro, född i Japan 1954, har tilldelats Nobelpriset i litteratur.}
\end{addmargin}

\begin{itemize}
    \item [\textbf{1.8}]
        Hur många namnentiteter har analysen hittat? Vilka typer och undertyper har de? Finns det någon entitet som inte är ett namnuttryck?

    \item [\textbf{1.9}]
        Vilka ord har fått positiva/negativa/neutrala attityder? 
        \textit{Tips: när du håller muspekaren över attitydsiffran 
        får du veta hur den ska tolkas.}
\end{itemize}

\section*{Övning 2 - 1800-talssvenska}
Sparv har ett analysläge för att annotera svenska texter från 1800-talet. Här används information från två äldre ordböcker (Dalin och Swedberg) för att få fram bättre analyser av ord med gammal stavning.

Gå till analysspråksmenyn och välj \textbf{svenska-1800-tal}. % \verb|svenska-1800-tal|
Ladda exemplet genom att trycka på Aftonbladet-knappen och kör analysen. Gå ner till meningen som börjar på ``Lådan var sannolikt för liten...''. I kolumnerna \textbf{lemma} och \textbf{lex} är det markerat vilket lexikon analysen kommer ifrån. Ordet \emph{liten} t.ex. får tre analyser eftersom ordet förekommer i både Dalin, Swedberg och SALDO.

\begin{itemize}
    \item [\textbf{2.1}] Från vilket lexikon får ordet \emph{blifva} sin analys i den ovanstående meningen?
    \item [\textbf{2.2}] Vad får man för information om man klickar på ordet i \textbf{lex}-kolumnen?
\end{itemize}

\section*{Övning 3 - Andra språk}
I nuläget har Sparv stöd för 20 analysspråk.
De flesta analystyper finns bara för svenska.
De andra språken har ordklasstaggning och lemmatisering.

\begin{itemize}
    \item [\textbf{3.1}]
        Välj ett språk från analysspråksmenyn, exempelvis finska, ryska eller latin.
        Skriv antingen in en egen text eller ladda ett exempel med
        knappen ovanför textfältet till höger.
        Gör nu en Sparvanalys för att få grundformer,
        en morfologisk analys och ordklasser. Ordklasser kommer
        från UDs taggmängd\footnote{\url{http://universaldependencies.org/u/pos/}}, men språken skiljer sig lite emellanåt
        i hur den morfologiska analysen presenteras.
\end{itemize}

\pagebreak

\section*{Övning 4 - Filuppladdning (avancerad)}
Istället för att mata in text i Sparvs textfält kan man använda sig av filuppladdningen
genom att trycka på \textbf{Ladda upp}-knappen. Accepterade filformat är txt (ren text) och xml.
Dokument skapade i Microsoft Word och liknande kan inte analyseras.
Om man laddar upp en xml-fil behöver man justera de avancerade inställningarna för att berätta
för Sparv vilka funktioner de olika xml-taggarna i indatan fyller. Om t.ex. dokumentelementet heter \emph{korpus}
så skriver man in det i textrutan bredvid \textbf{Dokumentelement} > \textbf{Tagg}. Attribut som inte
specifieras kommer att försvinna i resultatet. Man kan få en kort förklaring för en inställning
genom att klicka på frågetecknet bredvid den.
Observera att man kan ladda upp och analysera flera filer åt gången.
Inställningarna gäller då alla filer som laddas upp i samma körning.

\begin{itemize}
    \item [\textbf{4.1}]
		Testa att ladda upp en txt- eller xml-fil. Om du vill kan du skriva in din mailadress
		i email-fältet för att få ett mail med en nedladdningslänk när körningen är klar.
		Om du väljer att ladda upp en xml-fil får du kontrollera så att inställningarna
		innehåller rätt information om dina xml-taggar.
	\item [\textbf{4.2}]
		När körningen är klar ladda ner zip-filen genom att klicka på filsymbolen under
		\textbf{Färdiga körningar} eller genom att klicka på länken i mailet du har fått.
		Öppna zip-filen och xml-filen som ligger inuti. Har styckes- och meningsindelningen
		blivit korrekt? Har något av orden fått en sammansättningsanalys?
\end{itemize}

\end{document}
